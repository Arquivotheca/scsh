%&latex -*- latex -*-

%%%%%%%%%%%%%%%%%%%%%%%%%%%%%%%%%%%%%%%%%%%%%%%%%%%%%%%%%%%%%%%%%%%%%%%%%%%%%%%
\chapter{Introduction}

This is the reference manual for scsh, 
a {\Unix} shell that is embedded within {\Scheme}.
Scsh is a Scheme system designed for writing useful standalone Unix
programs and shell scripts---it spans a wide range of application,
from ``script'' applications usually handled with perl or sh,
to more standard systems applications usually written in C.

Scsh comes built on top of {\scm}, and has two components:
a process notation for running programs and setting up pipelines
and redirections, 
and a complete syscall library for low-level access to the operating system.
This manual gives a complete description of scsh.
A general discussion of the design principles behind scsh can be found
in a companion paper, ``A Scheme Shell.''

%%%%%%%%%%%%%%%%%%%%%%%%%%%%%%%%%%%%%%%%%%%%%%%%%%%%%%%%%%%%%%%%%%%%%%%%%%%%%%%
\section{Copyright \& source-code license}
Scsh is open source. The complete sources come with the standard
distribution, which can be downloaded off the net.
Scsh has an ideologically hip, BSD-style license.

We note that the code is a rich source for other Scheme implementations
to mine. Not only the \emph{code}, but the \emph{APIs} are available
for implementors working on Scheme environments for systems programming.
These APIs represent years of work, and should provide a big head-start
on any related effort. (Just don't call it ``scsh,'' unless it's
\emph{exactly} compliant with the scsh interfaces.)

Take all the code you like; we'll just write more.


%%%%%%%%%%%%%%%%%%%%%%%%%%%%%%%%%%%%%%%%%%%%%%%%%%%%%%%%%%%%%%%%%%%%%%%%%%%%%%%
\section{Obtaining scsh}
Scsh is distributed via net publication.
We place new releases at well-known network sites,
and allow them to propagate from there.
We currently release scsh to the following Internet sites:
\begin{inset}\begin{flushleft}
\ex{\urlh{ftp://ftp.scsh.net/pub/scsh}{ftp://ftp.scsh.net/pub/scsh}} \\
\ex{\urlh{http://prdownloads.sourceforge.net/scsh/}{http://prdownloads.sourceforge.net/scsh/}} \\
\end{flushleft}
\end{inset}
%
Each should have a compressed tar file of the entire scsh release,
which includes all the source code and the manual, 
and a separate file containing just this manual in Postscript form,
for those who simply wish to read about the system.

However, nothing is certain for long on the Net.
Probably the best way to get a copy of scsh is to use a network 
resource-discovery tool, such as archie, 
to find ftp servers storing scsh tar files.
Take the set of sites storing the most recent release of scsh,
choose one close to your site, and download the tar file.

\section{Building scsh}
Scsh currently runs on a fairly large set of Unix systems, including
Linux, FreeBSD, OpenBSD, NetBSD, MacOS X, SunOS, Solaris, AIX, NeXTSTEP, Irix, and HP-UX.
We use the Gnu project's autoconfig tool to generate self-configuring
shell scripts that customise the scsh Makefile for different OS variants.
This means that if you use one of the common Unix implementations,
building scsh should require exactly the following steps:
\begin{inset}
\begin{tabular}{l@{\qquad}l}
\ex{gunzip scsh.tar.gz} &       \emph{Uncompress the release tar file.} \\
\ex{untar xfv scsh.tar} &       \emph{Unpack the source code.}          \\
\ex{cd scsh-0.6.x} &              \emph{Move to the source directory.}    \\
\ex{./configure} &              \emph{Examine host; build Makefile.}    \\
\ex{make} &                     \emph{Build system.}
\end{tabular}
\end{inset}
When you are done, you should have a virtual machine compiled in
file \ex{scshvm}, and a heap image in file \ex{scsh/scsh.image}.
Typing
\begin{code}
make install
\end{code}
will install these programs in your installation directory
(by default, \ex{/usr/local}), along with a small stub startup
binary, \ex{scsh}.

If you don't have the patience to do this, you can start up
a Scheme shell immediately after the initial make by simply
saying
\codex{./scshvm -o ./scshvm -i scsh/scsh.image}
See chapter~\ref{chapt:running} for full details on installation
locations and startup options.

It is not too difficult to port scsh to another Unix platform if your
OS is not supported by the current release.
See the release notes for more details on how to do this.

%%%%%%%%%%%%%%%%%%%%%%%%%%%%%%%%%%%%%%%%%%%%%%%%%%%%%%%%%%%%%%%%%%%%%%%%%%%%%%%
\section{Caveats}

It is important to note what scsh is \emph{not}, as well as what it is.
Scsh, in the current release, is primarily designed for the writing of
shell scripts---programming.
It is not a very comfortable system for interactive command use: 
the current release lacks job control, command-line editing, a terse,
convenient command syntax, and it does not read in an initialisation
file analogous to \ex{.login} or \ex{.profile}.
We hope to address all of these issues in future releases;
we even have designs for several of these features;
but the system as-released does not currently provide these features.

%%%%%%%%%%%%%%%%%%%%%%%%%%%%%%%%%%%%%%%%%%%%%%%%%%%%%%%%%%%%%%%%%%%%%%%%%%%%%%%
\section{Naming conventions}
Scsh follows a general naming scheme that consistently employs a set of
abbreviations.
This is intended to make it easier to remember the names of things.
Some of the common ones are:
\begin{description}
\item [\ex{fdes}] 
        Means ``file descriptor,'' a small integer used in {\Unix}
        to represent I/O channels.

\item [\ex{\ldots*}]
        A given bit of functionality sometimes comes in two related forms, 
        the first being a \emph{special form} that contains a body of 
        {\Scheme} code to be executed in some context, 
        and the other being a \emph{procedure} that takes a procedural
        argument (a ``thunk'') to be called in the same context.
        The procedure variant is named by taking the name of the special form,
        and appending an asterisk. For example:
\begin{code}
;;; Special form:
(with-cwd "/etc"
  (for-each print-file (directory-files))
  (display "All done"))

;;; Procedure:
(with-cwd* "/etc"
  (lambda ()
    (for-each print-file (directory-files))
    (display "All done")))\end{code}

\item [\ex{\var{action}/\var{modifier}}]
    The infix ``\ex{/}'' is pronounced ``with,'' as in 
    \ex{exec/env}---``exec with environment.''

\item [\ex{call/\ldots}] 
        Procedures that call their argument on some computed value
        are usually named ``\ex{call/\ldots},'' \eg,
        \ex{(call/fdes \var{port} \var{proc})}, which calls \var{proc}
        on \var{port}'s file descriptor, returning whatever \var{proc}
        returns. The abbreviated name means ``call with file descriptor.''

\item [\ex{with-\ldots}]
        Procedures that call their argument, and special forms that execute
        their bodies in some special dynamic context frequently have
        names of the form \ex{with-\ldots}. For example, 
        \ex{(with-env \var{env} \vari{body}1 \ldots)} and 
        \ex{(with-env* \var{env} \var{thunk})}. These forms set
        the process environment body, execute their body or thunk,
        and then return after resetting the environment to its original
        state. 

\item[\ex{create-}]
        Procedures that create objects in the file system (files, directories,
        temp files, fifos, \etc), begin with \ex{create-\ldots}.

\item [\ex{delete-}]
        Procedures that delete objects from the file system (files, 
        directories, temp files, fifos, \etc), begin with \ex{delete-\ldots}.

\item[ \ex{\var{record}:\var{field}} ]
        Procedures that access fields of a record are usually written
        with a colon between the name of the record and the name of the
        field, as in \ex{user-info:home-dir}.

\item[\ex{\%\ldots}]
        A percent sign is used to prefix lower-level scsh primitives
        that are not commonly used.

\item[\ex{-info}]
        Data structures packaging up information about various OS
        entities frequently end in \ldots\ex{-info}. Examples:
        \ex{user-info}, \ex{file-info}, \ex{group-info}, and \ex{host-info}.

\end{description}
%
Enumerated constants from some set \var{s} are usually named
\ex{\var{s}/\vari{const}1}, \ex{\var{s}/\vari{const}2}, \ldots.
For example, the various {\Unix} signal integers have the names
\ex{signal/cont}, \ex{signal/kill}, \ex{signal/int}, \ex{signal/hup}, 
and so forth.

%%%%%%%%%%%%%%%%%%%%%%%%%%%%%%%%%%%%%%%%%%%%%%%%%%%%%%%%%%%%%%%%%%%%%%%%%%%%%%%
\section{Lexical issues}
Scsh's lexical syntax is just {\RnRS} {\Scheme}, with the following
exceptions.

\subsection{Extended symbol syntax}
Scsh's symbol syntax differs from {\RnRS} {\Scheme} in the following ways:
\begin{itemize}
\item In scsh, symbol case is preserved by \ex{read} and is significant on
      symbol comparison. This means 
      \codex{(run (less Readme))}
      displays the right file.

\item ``\ex{-}'' and ``\ex{+}'' are allowed to begin symbols. 
      So the following are legitimate symbols:
      \codex{-O2 -geometry +Wn}

\item ``\ex{|}'' and ``\ex{.}'' are symbol constituents.
  This allows \ex{|} for the pipe symbol, and \ex{..} for the parent-directory
  symbol. (Of course, ``\ex{.}'' alone is not a symbol, but a 
  dotted-pair marker.)

\item A symbol may begin with a digit.
      So the following are legitimate symbols:
\codex{9x15 80x36-3+440}
\end{itemize}

\subsection{Extended string syntax}
Scsh strings are allowed to contain the {\Ansi} C escape sequences
      such as \verb|\n| and \verb|\161|.
        
\subsection{Block comments and executable interpreter-triggers}
Scsh allows source files to begin with a header of the form
\codex{\#!/usr/local/bin/scsh -s}
The Unix operating system treats source files beginning with the headers
of this form specially; 
they can be directly executed by the operating system
(see chapter~\ref{chapt:running} for information on how to use this feature).
The scsh interpreter ignores this special header by treating \ex{\#!} as a
comment marker similar to \ex{;}. 
When the scsh reader encounters \ex{\#!}, it skips characters until it finds
the closing sequence
new\-line/{\ob}ex\-cla\-ma\-tion-{\ob}point/{\ob}sharp-{\ob}sign/{\ob}new\-line.

Although the form of the \ex{\#!} read-macro was chosen to support
interpreter-triggers for executable Unix scripts,
it is a general block-comment sequence and can be used as such
anywhere in a scsh program.

\subsection{Here-strings}
The read macro  \ex{\#<} is used to introduce ``here-strings''
in programs, similar to the \ex{<<} ``here document'' redirections
provided by sh and csh.
There are two kinds of here-string, character-delimited and line-delimited;
they are both introduced by the \ex{\#<} sequence.

\subsubsection{Character-delimited here-strings}
A \emph{character-delimited} here-string has the form 
\codex{\#<\emph{x}...stuff...\emph{x}}
where \emph{x} is any single character 
(except \ex{<}, see below),
which is used to delimit the string bounds.
Some examples:
\begin{inset}
\begin{tabular}{ll}
        Here-string syntax              & Ordinary string syntax \\ \hline
        \verb:#<|Hello, world.|:        & \verb:"Hello, world.":        \\
        \verb:#<!"Ouch," he said.!:     & \verb:"\"Ouch,\" he said.":
\end{tabular}
\end{inset}
%
There is no interpretation of characters within the here-string;
the characters are all copied verbatim.

\subsubsection{Line-delimited here-strings}
If the sequence begins "\ex{\#<<}" then it introduces a \emph{line-delimited}
here-string.
These are similar to the ``here documents'' of sh and csh.
Line-delimited here-strings are delimited by the rest of the text line that
follows the "\ex{\#<<}" sequence. 
For example:

\begin{code}
#<<FOO
Hello, there.
This is read by Scheme as a string,
terminated by the first occurrence
of newline-F-O-O-newline or newline-F-O-O-eof.
FOO\end{code}
%
Thus, 
\begin{code}
#<<foo
Hello, world.
foo\end{code}
%
is the same thing as
\codex{"Hello, world."}

Line-delimited here-strings are useful for writing down long, constant
strings---such as long, multi-line \ex{format} strings, 
or arguments to Unix programs, \eg,
\begin{code}
;; Free up some disk space for my netnews files.
(run (csh -c #<<EOF
cd /urops
rm -rf *
echo All done.

EOF
))\end{code}

The advantage they have over the double-quote syntax 
(\eg, \ex{"Hello, world."})
is that there is no need to backslash-quote special characters internal
to the string, such as the double-quote or backslash characters.

The detailed syntax of line-delimited here-strings is as follows.
The characters "\ex{\#<<}" begin the here-string.  
The characters between the "\ex{\#<<}" and the next newline are the
\emph{delimiter line}. 
All characters between the "\ex{\#<<}" and the next newline comprise the
delimiter line---including any white space. 
The body of the string begins on the following line,
and is terminated by a line of text which exactly matches the
delimiter line.  
This terminating line can be ended by either a newline or end-of-file.
Absolutely no interpretation is done on the input string.  
Control characters, white space, quotes, backslash---everything
is copied as-is.  
The newline immediately preceding the terminating delimiter line is
not included in the result string
(leave an extra blank line if you need to put a final
newline in the here-string---see the example above).  
If EOF is encountered before reading the end of the here-string, 
an error is signalled.

\subsection{Dot}
It is unfortunate that the single-dot token, ``\ex{.}'', is both
a fundamental {\Unix} file name and a deep, primitive syntactic token
in {\Scheme}---it means the following will not parse correctly in scsh:
\codex{(run/strings (find . -name *.c -print))}
You must instead quote the dot: 
\codex{(run/strings (find "." -name *.c -print))}

When you write shell scripts that manipulate the file system,
keep in mind the special status of the dot token.

%%%%%%%%%%%%%%%%%%%%%%%%%%%%%%%%%%%%%%%%%%%%%%%%%%%%%%%%%%%%%%%%%%%%%%%%%%%%%%%
\section{Record types and the \texttt{define-record} form}
\label{sec:defrec}
\index{define-record@\texttt{define-record}}

Scsh's interfaces occasionally provide data in structured record types;
an example is the \ex{file-info} record whose various fields describe the size,
protection, last date of modification, and other pertinent data for a
particular file.
These record types are described in this manual using the \ex{define-record}
notation, which looks like the following:
%
\begin{code}
(define-record ship
  x
  y
  (size 100))\end{code}
%
This form defines a \var{ship} record, with three fields: 
its x and y coordinates, and its size.
The values of the \var{x} and \var{y} fields are specified as parameters
to the ship-building procedure, \ex{(make-ship \var{x} \var{y})}, 
and the \var{size} field is initialised to 100.
All told, the \ex{define-record} form above defines the following procedures:
%
\begin{center}
\begin{tabular}{|ll|}
\multicolumn{1}{l}{Procedure} & \multicolumn{1}{l}{Definition} \\
\hline
(make-ship \var{x} \var{y}) & Create a new \var{ship} record. \\
\hline
(ship:x \var{ship})     & Retrieve the \var{x} field. \\
(ship:y \var{ship})     & Retrieve the \var{y} field. \\
(ship:size \var{ship})  & Retrieve the \var{size} field. \\
\hline
(set-ship:x \var{ship} \var{new-x}) & Assign the \var{x} field. \\
(set-ship:y \var{ship} \var{new-y}) & Assign the \var{y} field. \\
(set-ship:size \var{ship} \var{new-size}) & Assign the \var{size} field. \\
\hline
(modify-ship:x \var{ship} \var{xfun}) & Modify \var{x} field with \var{xfun}. \\
(modify-ship:y \var{ship} \var{yfun}) & Modify \var{y} field with \var{yfun}. \\
(modify-ship:size \var{ship} \var{sizefun}) & Modify \var{size} field with \var{sizefun}. \\
\hline
(ship? \var{object})    & Type predicate. \\
\hline
(copy-ship \var{ship}) & Shallow-copy of the record. \\
\hline
\end{tabular}
\end{center}
%

An implementation of \ex{define-record} is available as a macro for Scheme
programmers to define their own record types; 
the syntax is accessed by opening the package \ex{defrec-package}, which
exports the single syntax form \ex{define-record}. 
See the source code for the \ex{defrec-package} module 
for further details of the macro.

You must open this package to access the form.
Scsh does not export a record-definition package by default as there are
several from which to choose.
Besides the \ex{define-record} macro, which Shivers prefers\footnote{He wrote
it.}, you might instead wish to employ the notationally-distinct
\ex{define-record-type} macro that Jonathan Rees 
prefers\footnote{He wrote it.}.
It can be found in the 
\ex{define-record-types} structure.

Alternatively, you may define your own, of course.


%%%%%%%%%%%%%%%%%%%%%%%%%%%%%%%%%%%%%%%%%%%%%%%%%%%%%%%%%%%%%%%%%%%%%%%%%%%%%%%
\section{A word about {\Unix} standards}
``The wonderful thing about {\Unix} standards is that there are so many
to choose from.''
You may be totally bewildered about the multitude of various standards that
exist.
Rest assured that nowhere in this manual will you encounter an attempt
to spell it all out for you;
you could not read and internalise such a twisted account without
bleeding from the nose and ears.

However, you might keep in mind the following simple fact: of all the
standards, {\Posix} is the least common denominator.
So when this manual repeatedly refers to {\Posix}, the point is ``the
thing we are describing should be portable just about anywhere.''
Scsh sticks to {\Posix} when at all possible; its major departure is
symbolic links, which aren't in {\Posix} (see---it 
really \emph{is} a least common denominator).

%%% Local Variables: 
%%% mode: latex
%%% TeX-master: "man"
%%% End: 
